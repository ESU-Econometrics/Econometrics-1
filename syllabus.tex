\documentclass[]{tufte-handout}

% ams
\usepackage{amssymb,amsmath}

\usepackage{ifxetex,ifluatex}
\usepackage{fixltx2e} % provides \textsubscript
\ifnum 0\ifxetex 1\fi\ifluatex 1\fi=0 % if pdftex
  \usepackage[T1]{fontenc}
  \usepackage[utf8]{inputenc}
\else % if luatex or xelatex
  \makeatletter
  \@ifpackageloaded{fontspec}{}{\usepackage{fontspec}}
  \makeatother
  \defaultfontfeatures{Ligatures=TeX,Scale=MatchLowercase}
  \makeatletter
  \@ifpackageloaded{soul}{
     \renewcommand\allcapsspacing[1]{{\addfontfeature{LetterSpace=15}#1}}
     \renewcommand\smallcapsspacing[1]{{\addfontfeature{LetterSpace=10}#1}}
   }{}
  \makeatother

\fi

% graphix
\usepackage{graphicx}
\setkeys{Gin}{width=\linewidth,totalheight=\textheight,keepaspectratio}

% booktabs
\usepackage{booktabs}

% url
\usepackage{url}

% hyperref
\usepackage{hyperref}

% units.
\usepackage{units}


\setcounter{secnumdepth}{-1}

% citations

% pandoc syntax highlighting

% longtable

% multiplecol
\usepackage{multicol}

% strikeout
\usepackage[normalem]{ulem}

% morefloats
\usepackage{morefloats}


% tightlist macro required by pandoc >= 1.14
\providecommand{\tightlist}{%
  \setlength{\itemsep}{0pt}\setlength{\parskip}{0pt}}

% title / author / date
\title{EC 710 Syllabus}
\author{Dr.~Bekah Selby}
\date{Summer 2019}


\begin{document}

\maketitle




\hypertarget{description}{%
\section{Description}\label{description}}

This course presents the theory and tools that applied economists use to
answer complex questions in microeconomics. This is one of two
econometrics courses that satisfy the requirements for the MS
Informatics - Quantitative Economics Concentration. This course is held
100\% online with the following structure:

\begin{itemize}
\tightlist
\item
  \textbf{Monday-Thursday}: Self-paced, module-based learning. You will
  complete modules by submitting assignments. Online XA section students
  must complete at least two (2) discussion board assignments weekly.
  This will count for the participation part of the grade.
\item
  \textbf{Friday}: A 1-hour live webinar with Professor Selby.
  Attendance is \textbf{required} for individuals enrolled in the
  on-campus section (A), which will count for the participation part of
  the grade. This time may be used for elaboration, tutoring, questions,
  and discussion.
\end{itemize}

\noindent The focus in this class will be to prepare students to examine
microeconomic data, formulate models, and provide analysis of real-world
phenomena.

\subsection*{Prerequisites}

A background in economics, statistics, and calculus is expected.

\subsection*{Graduate Textbook}

The (\textbf{Required}) textbook for this course will be:

\begin{itemize}
    \item ``Mostly Harmless Econometrics'' by Joshua D. Angrist and J{\"o}rn-Steffen Pischke
    ISBN 978-0691120355
    \item ``R Markdown: The Definitive Guide'' Yihui Xie, J. J. Allaire, Garrett Grolemund, \url{https://bookdown.org/yihui/rmarkdown/}
\end{itemize}

\noindent Other recommended supplements:

\begin{itemize}

    \item ``Mastering 'Metrics'' by Joshua D. Angrist and J{\"o}rn-Steffen Pischke
    ISBN 978-0691152844
    \item ``A Guide to Econometrics'' by Peter Kennedy ISBN  978-1405182577
\end{itemize}

\subsection*{Undergraduate Textbook}

The (\textbf{Required}) textbook for this course will be:

\begin{itemize}
    \item ``Mastering 'Metrics'' by Joshua D. Angrist and J{\"o}rn-Steffen Pischke
    ISBN 978-0691152844
\end{itemize}

\noindent Other recommended supplements:

\begin{itemize}

    \item ``Mostly Harmless Econometrics'' by Joshua D. Angrist and J{\"o}rn-Steffen Pischke
    ISBN 978-0691120355
    \item ``A Guide to Econometrics'' by Peter Kennedy ISBN  978-1405182577
\end{itemize}

\subsection*{Technology}
\begin{itemize}
    \item R \url{https://www.r-project.org/}
    \item R-Studio \url{https://www.rstudio.com/products/rstudio/download/}
\end{itemize}

\subsection*{Assessment}

Your grade in this course will be determined by achievement in the
following categories:

\begin{enumerate}
    \item Module Homework 
    \item Participation 
    \item Replication Projects
\end{enumerate}

\subsection*{Grade Criteria}
\begin{center}
    \begin{tabular}{cc}
        \textbf{Letter Grade} & \textbf{Overall Grade} \\ \hline 
        A & 89.50-100\% \\
        B & 79.50-89.49\% \\
        C & 69.50-79.49\% \\
        D & 59.50-69.49\% \\
        F & 0-59.49\% \\
    \end{tabular}
\end{center}

\subsection*{Policy on Late Work}

I do not accept late work in this class except under the most
extraordinary of circumstances. If you know you will be gone for an
assignment or class, please make arrangements to have the work done
prior to your absence. I will happily accept early submissions on any
due work.

\subsection*{Attendance}

Attendance is measured in participation during discussions, contact with
professor, and interactions with peers. As with most mathematically
intensive courses, it is vital for success. You are responsible for
obtaining all material in lectures, discussions, and assignments. You
are required to turn in all written work on time. Missing an assignment,
activity, or a paper will lower your grade.

\subsection*{Student Learning Outcomes}

Upon completion of this course, students will be able to do the
following:

\begin{enumerate}
    \item Apply the theory behind linear regression analysis, including interpretation of estimated coefficients and goodness of fit.
    \item Develop models appropriate for data and questions.
    \item Illustrate results from panel-data models in a professional and clear way.
    \item Utilize data-analysis software to perform summary analysis and linear regression analysis.
    \item Demonstrate mastery of econometric analysis in an academic research replications.
    
\end{enumerate}

\subsection*{Academic Dishonesty}

Academic dishonesty includes, but is not limited to, acts of cheating
and plagiarism. Such acts will not be tolerated in this course and will
be reported. Consequences for academic dishonesty include an automatic
fail (F) in the course.

\subsection*{Student Accommodations Statement}

Emporia State University will make reasonable accommodations for persons
with documented disabilities. Students need to contact the Director of
Student Accessibility and Support Services, Stephanie Adams, Plumb Hall
106, and the professor as early in the semester as possible to ensure
that classroom and academic accommodations are implemented in a timely
fashion. All communication between students, Student Accessibility and
Support Services, and the professor will be strictly confidential.

\subsection*{Diversity Statement }

Emporia State University supports an inclusive learning environment
where diversity and individual differences are understood, respected,
appreciated, and recognized as a source of strength. We expect that
students and faculty at Emporia State will respect differences and
demonstrate diligence in understanding how other people's perspectives,
behaviors, and world-views may be different from their own.

If there are aspects of the design, instruction, and/or your experiences
within this course that result in barriers to your inclusion or accurate
assessment of achievement, please notify the unit head (Department Chair
or equivalent) as soon as possible, and/or contact the office of the
Assistant Dean of Students for Diversity, Equity \& Inclusion.

\textbackslash{}end\{document\}



\end{document}
